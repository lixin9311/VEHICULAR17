\section{Related works} \label{sec:related_works}

In CITS, vehicles frequently broadcast a message containing its status to its surrounding ITS-Ss.
This message is standardized as CAM  in EU by ETSI.
CAMs are messages exchanged in the ITS network between ITS-Ss to create and maintain awareness of each other
and to support the cooperative performance of vehicles using the road network.
A CAM contains status and attributes information of the originating ITS-S.
The content varies depending on the type of the ITS-S.
For vehicles the status information includes time, position, motion state, activated systems,
etc. and the attribute information includes data about the dimensions,
vehicle type, and role in the road traffic, etc.
On reception of a CAM, the receiving ITS-S becomes aware of the presence, type, and status of the originating ITS-S.
The received information can be used by the receiving ITS-S to support several ITS applications.
For example, the CAM can represent the originating vehicle, upon receiving a CAM,
we usually store the information in Local Dynamic Map (LDM).

An LDM is a facility in CITS.
It stores the information of on road objects.
Such as surrounding vehicles and pedestrians or traffic lights.
The information maintained by LDM could be accessed by other ITS applications to assist the driver or facilitate autonomous driving.

Japan Metropolitan Police Department has developed Driving Safety Support Systems (DSSS) to reduce accident.
DSSS is an infrastructure-based system using FM broadcast and deployed to the highways in Japan.

In our previous work, we have proposed an infrastructure-bases Proxy CAM to overcome the unstable V2V communication under some environments.
We use roadside units (RSUs) to sense non-ITS objects, and generate CAMs for them, then broadcast to surrounding road users.
