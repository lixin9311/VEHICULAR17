\documentclass[conference,twocolumn]{IEEEtran}

\usepackage[utf8]{inputenc} % set input encoding (not needed with XeLaTeX)

\usepackage{graphicx} % support the \includegraphics command and options
\usepackage{booktabs} % for much better looking tables
\usepackage{subfig}
\usepackage{amsmath}

\begin{document}
% TODO: Better title
\title{Improved Proxy Cooperative Awareness Message: Mix of Infrastructure and Non-Infrastructure Assisted V2V Messaging}
\author{
  \IEEEauthorblockN{Xin Li, Ye Tao, Manabu Tsukada}
  \IEEEauthorblockA{Graduate School of Information Science and Technology, The University of Tokyo \\
  \{lucus, tydus, tsukada\}@hongo.wide.ad.jp}
}

\maketitle

\begin{abstract}
Cooperative Intelligent Transportation Systems
(CITS) are being developed to make road traffic safer and more efficient.
Vehicle to vehicle (V2V) communication plays an important role for the cooperative
awareness of vehicles in all the CITS standards in EU, US, and Japan.
Essentially the CITS application relies on the exchanged data by V2V messages (called Cooperative Awareness Message (CAM) in EU).
In our previous work, we have presented an infrastructure-based system to overcome the 2 significant problems:
1) mixed environment and 2) obstacle interference.
But the previous design still relies on infrastructure, hence it is still constrained by the environment.
Moreover, large vehicles on road could be obstacles for sensors and even wireless communications.
To solve these problems, we propose the system called Improved Proxy CAM.
It could be installed on large vehicles and infrastructure to generate or relay the necessary V2V messages.
We design the system based on standardized ITS architecture in ISO/ETSI, and it could be easily adapted to other ITS architectures.
\end{abstract}

\begin{IEEEkeywords}
 Keyword 1, Keyword 2, Keyword 3
\end{IEEEkeywords}

\section{Introduction} \label{sec:intro}

Road transportation is an essential part of human daily life.
Intelligent Transportation Systems (ITS) is designed to improve the road traffic by achieving safe,
efficient and autonomous transportation.
Cooperative ITS and vehicular communication are found essential to the
cooperation of on-road vehicles and other forms of transport and pedestrians,
whereby sensor information is shared between road users to,
in effect, extend a driver’s field of vision.
Cooperative ITS can not only be used to provide driver assistance with the not so distant future
but also to feed more data to full autonomous driving.

International Organization for Standardization (ISO) Technical Committee 204 (ISO / TC 204)
is responsible for the overall system aspects and infrastructure aspects of intelligent transport systems (ITS).
And Working Group 18 (ISO/TC 204/WG 18) of it is responsible for Cooperative systems.
Also in Europe, the European Telecommunications Standards Institute (ETSI) has standardized wireless communication and applications of ITS.
In the US, the Institute of Electrical and Electronics Engineers (IEEE) has standardized Wireless Access
in Vehicular Environments (WAVE) architecture in IEEE 1609 family of standards and IEEE802.11p.

ETSI has standardized a Cooperative Awareness (CA) Basic Service to inform each road user about other's position, dynamics, and attributes.
The awareness of each other is the basis for several road safety and traffic efficiency applications.
It is achieved by regular exchange of information among vehicles and between vehicles (V2V)
and road side infrastructure (V2I and I2V) based on wireless networks.
Other CITS architectures besides ETSI standard have similar designs to provide such function.
The information to be exchanged for cooperative awareness is picked up in the periodically transmitted Cooperative Awareness Message (CAM).
This service is a mandatory facility for all kind of ITS-Stations (ITS-S).
The dissemination of CAMs may vary depending on the applied communication system.
But in most cases, CAMs are sent by the originating ITS-S to all ITS-Ss within the direct communication range,
and should not be relayed further.

Single hop broadcast is one of the fundamental techniques for Cooperative Awareness service in CTIS.
Thanks to it, drivers or autonomous vehicles can achieve a broader view in theory.
But in the realistic world, such systems could be seriously limited.
Deployment of ITS is hard, we cannot retire all current non-ITS vehicles at once.
Such non-ITS vehicles and road users could not send CAMs for themselves.
Despite the deployment, the messages could suffer interferences caused by other obstacles such as buildings and large vehicles.
Also, the view of sensors could be blocked.

In our previous work~\cite{kitazato2016proxy}, we have proposed a Proxy CAM system,
using roadside infrastructure to generate the necessary V2V messages for the non-ITS vehicles.
But such system is still limited by the environment, it requires infrastructure devices.
The objective of the paper is to improve the Proxy CAM by utilizing large vehicles or the obstacles themselves.
The proposed system, called Extended Proxy CAM,
generates CAMs for non-ITS vehicles and road users with installed sensors to sensing them.
Meanwhile, relay the CAMs sent from surrounding vehicles, when the large vehicles with such system could interfere the V2V communications.
Currently, such design is working in progress, we present a preliminary design in this paper.

The contributions of this work are:
\begin{itemize}
\item Analysis of potential interference of large vehicles
\item Proposal of platform for infrastructure-free assisted CA service
\end{itemize}

The rest of the paper is organized as follows.
Section~\ref{sec:related_works} highlights the related works.
Section~\ref{sec:problem} analyzes the issues of V2V communications and our previous work, Proxy CAM,
then summarizes the requirements of the solution.
Section~\ref{sec:iproxy_cam} presents the system design.
Finally, Section~\ref{sec:conclusion} concludes this paper.
\section{Section 2}

Here is the section 2.

\subsection{Subsection}

Here is the subsection.
\section{Section 3}

Here is the section 3.

\subsection{Subsection}

Here is the subsection.
\section{Improved Proxy CAM} \label{sec:iproxy_cam}

To solve the problem described in Section~\ref{sec:problem},
we extended our previous work.
The new design is called Extended Proxy CAM.
We want to make the system compatible with original CA service,
but due to some security reason, the compatibility is still under pursuit.
The new WIP design is based on the ETSI standard ITS,
it should be easy to apply this design to other ITS standards.
The current preliminary design is described in this section.

\subsection{System design}

Fig 1 shows the overview of the system design.
Our previous work is limited to RSU, but the new design should also be applicable to moving vehicles, such as large trucks.

ITS enabled vehicles are usually equipped with sensors to detect the pedestrian
and other vehicles or objects in the surrounding within the visual field.
Then record the data in LDM.
When the system is deployed on vehicles, we can utilize the LDM instead of harvesting data directly from sensors.

For those non-ITS enabled road users,
we can directly use the objects from LDM to generate CAMs for them.
Then the system broadcasts the generated CAMs to its surrounding ITS-Ss.
Upon receiving the CAMs, we can process the message just like normal CAMs,
storing the information into their own LDMs.

For those ITS enabled road users,
from which the CAMs could suffer serious loss due to the large vehicle with this system deployed,
things could be a little complex. First, upon receiving the CAM, we need to check the plausibility of it,
using similar techniques described in Section 2, to ensure the sender is not a malicious one.
Only if it passed the plausibility check, the intermediate vehicle, as shown in Fig 1,
should append its signature and public key to the CAM, then relay the CAM to its surrounding vehicles.

In the following sections, we describe the system in detail.
The whole system can be split into two part, CAM generator, and CA relay.

\subsubsection{CAM generator}

The CAM generator generates the necessary CAMs for those non-ITS enabled road users.
It is based on our previous work. Because our new design could be deployed to vehicles,
we could save sensors if the system is installed on a vehicle.
The system could use the data in LDM on the vehicle to provide more detailed information (lane position, car type, etc.).
It's easy to distinguish those non-ITS enabled vehicles in LDM.
Additionally, we can generate CAMs for other on road objects which could endanger the road traffic.
If we could identify the vehicle's ID from sensors,
we should assign an ID connected with it to the corresponding object in LDM.
Otherwise, a random ID is assigned.
Object tracking techniques should be used to ensure that we assign the same object with the same ID.

As shown in Fig 1, when the truck B recognize a walking man with its sensors,
which is apparently a non-ITS enabled road user,
it firstly records the data into its LDM and assigned a random ID to it.
Vehicle C cannot perceive the existence of this walking man because truck B blocks the view of its sensors.
Then the CAM encoder on truck B collect the information from its LDM to generate CAM for this walking man,
then broadcast the CAM to its surrounding vehicles.
Therefore, vehicle C knows there is a collision risk with the walking man if it overtakes truck B.
At the same time, with object tracking techniques,
truck B constantly tracks the walking man, and update the information in its LDM. 

\subsubsection{CA relay}

We use CA relay to overcome the issue described in Section 3,
large vehicles could interfere the V2V wireless communications, resulting in CAMs losses.
CA relay is used to relay the CAMs from surrounding vehicles to surrounding vehicles.
As shown in Fig 1,
the V2V communication between vehicle A and vehicle C could suffer serious loss due to the truck B blocking the signal.
Under such scenario, it could be dangerous if vehicle C decide to overtake truck B from behind.
We use the truck B as an intermediate station to relay the CAMs from vehicle A.
Upon truck B receiving CAM from vehicle A, it firstly conducts a plausibility check,
to ensure the CAM is not from a malicious sender.
If passed the check,
it should sign-off the CAM using its own private key and append its public key to the CAM then broadcast the signed CAM.

Upon receiving a relayed CAM, the receiver should firstly check the signature of the message,
then it should conduct a plausibility check if the intermediate node is a vehicle.
Only if the CAM passed all checks, we can consider the CAM is a valid one.

Such system should also be applicable to RSUs.
Fig 2 describes a scenario which is common in Japan,
in where some streets are narrow, and the buildings block the view at crossroads.
Drivers always have to stop at each crossroad to ensure it is safe to pass.
Even with ITS, the V2V communication still could be blocked by the buildings.
If we place some RSUs with our system on each side of the crossroad and connect them to each other to generate and relay the CAM,
the vehicle could have a better perception of other vehicles.

\subsection{Considerations}

Currently, this design is still not completed yet.
To overcome the issued and meet the requirements described in Section 3,
there are some considerations as follows.
\begin{enumerate}
\item When deployed on vehicles,
    the system should only be deployed those which could endanger the road safety and also under serious supervision,
    such as trucks, buses and etc.
    Because there is a risk, that an attacker could manipulate the CAM to attack the network,
    so it is not safe to deploy such system to normal vehicles.
\end{enumerate}
\section{Conclusion} \label{sec:conclusion}

In this paper, we have shown what issues are left unsolved with our previous design.
Then we have presented a preliminary design of Extended Proxy CAM,
to provide an infrastructure-free assisted V2V message system.
Such design is aimed to generate CAMs for those non-ITS enabled road users,
and meanwhile relay the CAMs when the CAMs could suffer interference.

We haven't implemented such system yet. For the future work, we are going to:
\begin{enumerate}
\item Implement an experimental system to do field test, to see how much such system can improve.
\item Discuss the security of such system, to improve its compatibility and make it proof to attacks.
\item Integrate it with existed ITS framworks or stacks, and make it a standard.
\end{enumerate}

\bibliographystyle{unsrt}
\bibliography{main}

\end{document}
