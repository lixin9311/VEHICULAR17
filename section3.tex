\section{Problem statement} \label{sec:problem}
In the last two section, we introduced the CITS standard and its limitation on the Cooperative Awareness subsystem.
Then we recalled our previous proposed Proxy CAM system employd on roadside units.
However this system design still left different issues unsolved.
This section will analyze the issues deeply and state additional requirements while solving the issues.

\subsection{Issues}
% Not working without RSU
In the original design of Proxy CAM system, only Road-side Units (RSUs) are permitted to transmit proxied CAM messages.
Thus, the largest limitation of the Proxy CAM system is the dependency on RSUs.

% Deployment
In practice, several shortcomings on deployment should be considered.
Firstly, percepting vehicles requires dedicated hardwares to be deployed on the RSUs: ITS routers, sensors including cameras, lidars, etc.
Additionally, deployment of these hardwares in different environment brings extra challenges.
To summarize, deployment difficulties could result in a huge stress to the relevant departments for both cost and other resources, and the cost of maintenance will increase as well.

% View coverage & Signal blocking
Moreover, even on the crossroads with Proxy CAM RSU deployed, preception and communication problems may still happen.
Large vehicles (buses, trucks, etc.) could block the view of sensor and even block 802.11 signal, which may lead to a false-negative perception and cause trouble.

We did a preliminary experiment on a highway in Tokyo.
TODO TODO TODO TODO TODO TODO TODO TODO TODO TODO


\subsection{Requirements}
The Proxy CAM system maintained the standard compatibility by sending the proxy CAM message in the destination vehicle's behavior.
Since the Proxy CAM system is only employed on managed Road-Side Units, the information security issues are not considered.

Contrastly, in the new system, the Proxy CAM generator could be deployed on the vehicles, thus the authoritativeness cannot be ensured.
This may lead to a dilemma: how to ensure the information security while maintain standard compatibility.

% Standard compatibility
TODO TODO TODO TODO TODO TODO TODO TODO TODO TODO

% Information Security
The Extended Proxy CAM requires some vehicles along with the RSUs to send the Proxy CAM message, i.e. mocking another vehicle.
For an instance, a malicious node could easily insert ghost vehicles on other vehicles' LDM by sending fake Proxy CAM messages, which could cause serious security crisis.

ITS standard defines a PKI-based trust model. However it relies on certification revocation, thus could not take effect in real-time.
A real-time trust model should be developed for the Extended Proxy CAM system to ensure the potential attacker could not easily break the system.

